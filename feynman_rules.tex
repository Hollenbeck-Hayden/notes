\documentclass[12pt]{article}
\usepackage[letterpaper, portrait, margin=0.5in]{geometry}
\usepackage[utf8]{inputenc}

\usepackage{amsmath}
\usepackage{amsthm}
\usepackage{amssymb}
\usepackage{braket}
\usepackage{bm}
\usepackage{slashed}
\usepackage{simpler-wick}

\usepackage{hyperref}
\hypersetup{
    colorlinks,
    citecolor=blue,
    filecolor=blue,
    linkcolor=blue,
    urlcolor=blue
}

\setlength{\parskip}{1em}

\newcommand{\diff}{\mathop{}\!\mathrm{d}}
\newcommand{\R}{\mathbb{R}}
\newcommand{\C}{\mathbb{C}}

\newcommand{\normord}[1]{:\mathrel{#1}:}

\theoremstyle{definition}
\newtheorem{defn}{Defn}[section]
\newtheorem{thm}{Theorem}[section]
\newtheorem{corl}{Corollary}[section]

\newenvironment{problem}[1]
    {\noindent
    \textbf{#1})
    }
    {
    \vspace{12pt}
    }

\begin{document}

\tableofcontents

\pagebreak

\section{General Rules}

\subsection{Feynamn Rule Procedure}

Used to calculate the matrix scattering element from a diagram:
\begin{equation*}
    -i \mathcal{M}
\end{equation*}

\textbf{1. Momentum Conservation} At each vertex, there's a momentum conserving delta function
\begin{equation*}
    (2\pi)^4 \delta^{(4)} \left( \sum_i p_i \right) 
\end{equation*}
where the sign of $p_i$ is determined by momentum arrows on the diagram.

\textbf{2. Internal Momenta} For each internal momenta $q$, integrate over it
\begin{equation*}
    \int \frac{\diff^4 q}{(2\pi)^4}
\end{equation*}

\textbf{3. Drop overall delta-function} There will be an overall momentum conserving delta function of
incoming and outgoing momenta:
\begin{equation*}
    (2\pi)^4 \delta^{(4)}(p_{in} - p_{out})
\end{equation*}
just drop it.

\textbf{4. Divide by symmetry factor} Divide by the symmetry factor of the diagram.

\subsection{Unorganized Identies}

\subsubsection{Delta Functions}
\textbf{Delta-Function of Zero}: Let $V$ be the finite volume of space and $T$ be the finite duration of time.
Then
\begin{equation*}
\begin{split}
    2\pi \delta(0)
        &= \int_{-T/2}^{T/2} e^{-i(0) t} \diff t = \int_{-T/2}^{T/2} \diff t = T \\
    (2\pi)^3 \delta^3(0) &= V \\
\end{split}
\end{equation*}

\subsubsection{Lorentz Invariant Measure}
\begin{equation*}
    \int \frac{\diff^3 p}{(2\pi)^3 2 \omega_p} = \int \frac{\diff^4 p}{(2\pi)^4} \theta(p^0) \delta(k^2 - m^2)
\end{equation*}

\subsection{Characterization Techniques}
Physical quantities of a field solution may be calculated from the stress-energy tensor derived from the
field's Lagrangian via Noether's theorem. For example, in scalar field theory, the tensor, energy, and
momenta are given by:
\begin{equation*}
\begin{split}
    T^{\mu\nu} &= \frac{\partial \mathcal{L}}{\partial (\partial_\mu \phi)}  \partial^\nu \phi - \mathcal{L} g^{\mu\nu} \\
    H &= \int T^{00} \diff^3 x = \int (\pi \dot{\phi} - \mathcal{L}) \diff^3 x \\
    P^i &= \int T^{0i} \diff^3 x = - \int \pi \partial_i \phi \diff^3 x \\
\end{split}
\end{equation*}
If we're unsure the physical properties of a given solution, we may run it through the above machinery to
explicitly determine its behavior. Generally we'll have some solution as the eigenvalue of the corresponding
differential operator (e.g. $\omega_p$ for the Hamiltonian $i\partial_t$ or $\bm{k}$ for the momentum $i\nabla$)
times the number operator in that Fourier mode (e.g. $a^\dagger_k a_k$).

\textbf{Charge}: For $U(1)$ gauge symmetries

\subsection{Second Quantization}
\begin{enumerate}
    \item Classical Lagrangian density: all fields are classical functions
        \begin{equation*}
            \mathcal{L} = \frac{1}{2} \partial_\mu \phi \partial^\mu \phi - \frac{1}{2} m^2 \phi^2 - \frac{\lambda}{4!}\phi^4
        \end{equation*}
    \item Classical Hamiltonian
        \begin{equation*}
        \begin{split}
            \pi = \frac{\partial \mathcal{L}}{\partial \dot{\phi}} \qquad \qquad
            H = \int (\pi \dot{\phi} - \mathcal{L}) \diff^3 x
        \end{split}
        \end{equation*}
    \item Quantization: promote $\phi$ and $\pi$ to operators. Impose canonical commutation relations
        \begin{equation*}
            [\phi(\bm{x}), \pi(\bm{y})] = \delta^3(\bm{x} - \bm{y})
        \end{equation*}
        Now the Hamiltonian $H$, when written in these operators, is the familiar quantum mechanical
        Hamiltonian.
\end{enumerate}

\subsection{Crossing Symmetry}
Consider the process $AB \to CD$ with momenta $p_A, p_B, p_C$, and $p_D$. The matrix element is
\begin{equation*}
    \mathcal{M}_{AB \to CD} (p_A, p_B, p_C, p_D)
\end{equation*}
We may use charge conjugation to swap initial and final state particles. Using the Feynman-Stuckelberg
interpretation of antiparticles, this will produce the same matrix element but with opposite momentum.

\textbf{Example: }
$A\overline{C} \to \overline{B}D$
\begin{equation*}
\begin{split}
    \mathcal{M}_{AB \to CD}(p_A, p_B, p_C, p_D) &= \mathcal{M}_{A\overline{C} \to \overline{B}D}(p_A, -p_C, -p_B, p_D) \\
    p_B &\leftrightarrow -p_C \\
    s = (p_A + p_B)^2 &\to (p_A - p_C)^2 = t \\
    t = (p_A - p_C)^2 &\to (p_A + p_B)^2 = s \\
    u = (p_A - p_D)^2 &\to (p_A - p_D)^2 = u \\
\end{split}
\end{equation*}
Crossing symmetry lets us reuse matrix element calculations by just swapping Mandelstam variables.

\subsection{Classical Lagrangian and Hamiltonian Field Theory}

\section{Fermions}

\subsection{Gamma Matrices}

\textbf{Dirac Representation}:

\textbf{Chiral Representation}:

\textbf{Conjugation Identities}:
\begin{equation*}
    \gamma^0 (\gamma^\mu)^\dagger \gamma^0 = \gamma^\mu \\
\end{equation*}


\subsection{Spinors}

\textbf{Notation}:
\begin{itemize}
    \item General Dirac 4-spinors are $\psi$
    \item Positive solutions to the Dirac equation are labelled $u^{(s)},v^{(s)}$ for $s = 1,2$. The
        solutions $u$ are particles and $v$ are antiparticles.
    \item Positive and negative solutions to the Dirac equation are labelled $u^{(i)}$ for $i=1,2,3,4$.
\end{itemize}

Dirac 4-spinor positive energy solutions:
\begin{equation*}
    u = N
    \begin{pmatrix}
        \phi \\
        \frac{\bm{\sigma} \cdot \bm{p}}{E + m} \phi \\
    \end{pmatrix}
    \qquad \qquad
    v = N
    \begin{pmatrix}
        \frac{\bm{\sigma} \cdot \bm{p}}{E + m} \chi \\
        \chi \\
    \end{pmatrix}
\end{equation*}
where $\phi,\chi$ are two spinors and $N = \sqrt{E + m}$ is a normalization. The negative energy solutions are
related to $v$ by
\begin{equation*}
    u^{(3)}(p) = -v^{(2)}(-p) \qquad \qquad u^{(4)}(p) = v^{(1)}(-p)
\end{equation*}

\textbf{Dirac conjugate}:
\begin{equation*}
    \overline{\psi} = \psi^\dagger \gamma^0 
\end{equation*}

\textbf{Orthonormality}:
\begin{equation*}
    \begin{array}{ll}
        \overline{u}^{(r)} u^{(s)} = 2m \delta^{rs} & \overline{u}^{(r)} v^{(s)} = 0 \\
        \overline{v}^{(r)} v^{(s)} = -2m \delta^{rs} & \overline{v}^{(r)} u^{(s)} = 0 \\
    \end{array}
\end{equation*}
This determines the normalization $N = \sqrt{E + m}$. 

\textbf{Completeness}:
\begin{equation*}
\begin{split}
    \sum_s u^{(s)} \overline{u}^{(s)} &= \slashed{p} + m \\
    \sum_s v^{(s)} \overline{v}^{(s)} &= \slashed{p} - m \\
\end{split}
\end{equation*}

\textbf{Conjugation of Scalar Product}
\begin{equation*}
\begin{split}
    \left(\overline{\psi}^{(r)} \gamma^\mu \psi^{(s)}\right)^\dagger
        &= \left( \psi^{(r)\dagger} \gamma^0 \gamma^\mu \psi^{(s)} \right)^\dagger \\
        &= \psi^{(s)\dagger} \gamma^{\mu,\dagger} \gamma^0 \psi^{(r)} \\
        &= \overline{\psi}^{(s)} \gamma^0 \gamma^{\mu,\dagger} \gamma^0 \psi^{(r)} \\
        &= \overline{\psi}^{(s)} \gamma^\mu \psi^{(r)} \\
\end{split}
\end{equation*}

\subsection{Helicity}
\textbf{Helicity}: The helicity operator is defined as
\begin{equation*}
    h = \frac{\hat{\bm{S}} \cdot \hat{\bm{p}}}{|\bm{p}|}
\end{equation*}
for spin-1/2 fermions, they have helicity $\pm \frac{1}{2}$, where right-handed is positive.

Massless fermions have definite helicity since their momentum points in the same direction in all frames,
whereas there's always a frame for massive fermions where the momentum reverses and thus have opposite helicity.
Massless fermions therefore may be decomposed into left and right handed Weyl spinors.

\subsection{Chirality}
\textbf{Handedness Projection Operators}:
\begin{equation*}
    P_L = \frac{1 - \gamma^5}{2} \qquad \qquad P_R = \frac{1 - \gamma^5}{2}
\end{equation*}
Identities (where $h = L,R$ and $\overline{h}$ is the opposite handedness of $h$):
\begin{equation*}
\begin{split}
    P_L + P_R &= 1 \\
    P_h^2 &= P_h \\
    \gamma^\mu P_h &= P_{\overline{h}} \gamma^\mu \\
    P_h^\dagger &= P_h \\
\end{split}
\end{equation*}
\begin{equation*}
\begin{split}
    u_L &= P_L u \\
    \overline{u} P_L &= u^\dagger \gamma^0 P_L \\
        &= u^\dagger P_R \gamma^0 \\
        &= u_R \gamma^0 = \overline{u}_R \\
\end{split}
\end{equation*}

\section{Scalar Field Theory}

\subsection{Real Fields}
\begin{equation*}
    \mathcal{L} = \frac{1}{2} (\partial_\mu \phi)(\partial^\mu \phi) - \frac{1}{2} m^2 \phi^2 - \frac{\mu}{3!} \phi^3 - \frac{\lambda}{4!} \phi^4
\end{equation*}

\textbf{Field Solution}:
\begin{equation*}
    \phi(\bm{x}) = \int \frac{\diff^3 p}{(2\pi)^3} \frac{1}{\sqrt{2\omega_p} } \left( a_p e^{-i \bm{p} \cdot \bm{x}} + a^\dagger_p e^{i \bm{p} \cdot \bm{x}} \right) \\
\end{equation*}

\textbf{1. External Lines} For amputated diagrams, they're just a factor of 1. Otherwise external lines are
propagators.

\textbf{2. Vertex Factors}:
\begin{center}
    \begin{tabular}{ccc}
        3-point & \qquad & 4-point \\
        \hline
        $-i\mu$ & & $-i\lambda$
    \end{tabular}
\end{center}

\textbf{3. Propagators}:
\begin{equation*}
    \tilde{D}(p) = \frac{i}{p^2 - m^2 + i\epsilon}
\end{equation*}

\textbf{Normalization}:
\begin{equation*}
\begin{split}
    \ket{\bm{p}} &= \sqrt{2\omega_p} a^\dagger_p \ket{0} \\
    \braket{\bm{p}|\bm{p}'} &= 2\omega_p (2\pi)^3 \delta^3(\bm{p} - \bm{p}') \\
    \braket{\bm{p}|\bm{p}'} &= 2\omega_p (2\pi)^3 \delta^3(0) = 2\omega_p V \\
\end{split}
\end{equation*}

\textbf{Creation Operators}:
Creation a particle with momentum $\bm{p}$ at position $x$:
\begin{equation*}
\begin{split}
    \phi(x) \ket{0} &= \int \frac{\diff^3 p}{(2\pi)^3 2\omega_p} e^{i p \cdot x} \ket{\bm{p}} \\
\end{split}
\end{equation*}

\section{Quantum Electrodynamics (QED)}
\textbf{1. External Lines} Following the arrows on a fermion line, start with the end of a fermion line
and work backwards to get the right ordering. You should always start with a conjugate spinor and end
with a spinor, thus making a scalar.

\begin{center}
    \begin{tabular}{cccc}
        & Real Fermions & Antifermions & Photons \\
        \hline
        Incoming & $u$ & $\overline{v}$ & $\epsilon_\mu$ \\
        Outgoing & $\overline{u}$ & $v$ & $\epsilon^*_\mu$ \\
    \end{tabular}
\end{center}

\textbf{2. Vertex factors}:
\begin{enumerate}
    \item Each factor has $i g_e \gamma^\mu$
    \item $g_e = - q' \sqrt{4 \pi \alpha}$ where $q'$ is the fractional electric charge of the fermion in units
        of $e$.
    \begin{enumerate}
        \item $g_e = \sqrt{4\pi\alpha} $ for electrons
        \item $g_e = -(2/3) \sqrt{4\pi\alpha}$ for up quarks
    \end{enumerate}
\end{enumerate}

\textbf{3. Propagators} with 4-momentum $q$ and mass $m$ (note $q$ is in general off-shell)

Fermion propagators
\begin{equation*}
    \frac{i(\slashed{q} + m)}{q^2 - m^2}
\end{equation*}

Photon propagators
\begin{equation*}
    \frac{-i g_{\mu\nu}}{q^2}
\end{equation*}


\section{Kinematics}

\subsection{Maximum Product Energy}
Consider a process in the CoM frame where we wish to determine the maximum energy a product could have.
\begin{equation*}
    \begin{pmatrix}
        E_0 \\ 0
    \end{pmatrix}
    =
    \begin{pmatrix}
        E \\ \bm{p}
    \end{pmatrix}
    + \sum_i
    \begin{pmatrix}
        E_i \\ \bm{p}_i
    \end{pmatrix}
\end{equation*}
where $i$ indexes the other products. By momentum conservation $\bm{p} = -\sum \bm{p}_i$. Boost to the
CoM frame of the products:
\begin{equation*}
\begin{split}
    \begin{pmatrix}
        E_{0,new} \\ \bm{p}_{0,new}
    \end{pmatrix}
        &=
            \begin{pmatrix}
                E_{new} \\ \bm{p}_{new}
            \end{pmatrix}
            + \sum_i
            \begin{pmatrix}
                E_{i,new} \\ \bm{p}_{i,new}
            \end{pmatrix} \\
        &=
             \begin{pmatrix}
                E_{new} \\ \bm{p}_{new}
            \end{pmatrix}
            +
            \begin{pmatrix}
                \sum_i E_i \\ 0
            \end{pmatrix}
\end{split}
\end{equation*}
Then $E_{new} = E_{0,new} - \sum E_{i,new}$ is minimized when there's no relative kinetic energy between the
other products and $E_{i,new} = m_i$. The invariant mass gives
\begin{equation*}
    \left( \sum_i m_i \right)^2 = (p')^2 = \left( \sum_i E_i \right)^2 - \bm{p}^2
\end{equation*}
So the original CoM energy conservation gives
\begin{equation*}
\begin{split}
    (E_0 - E)^2 &= \left( \sum_i E_i \right)^2 = \left( \sum_i m_i \right)^2 + \bm{p}^2 \\
    E_0^2 - 2E_0 E + m^2 + \bm{p}^2 &= \\
    E &= \frac{E_0^2 + m^2 - \left( \sum_i m_i \right)^2}{2E_0} \\
\end{split}
\end{equation*}

\subsection{2-Scattering in Center of Mass Frame}
Incoming particles 1 and 2; outgoing particles 3 and 4.
\begin{equation*}
\begin{split}
    E_i &= \sqrt{\bm{p}_i^2 + m_i^2} \\
    \bm{p} &= \bm{p}_1 = -\bm{p}_2 \\
    \bm{p}' &= \bm{p}_3 = - \bm{p}_4 \\
    \bm{p} \cdot \bm{p}' &= |\bm{p}| |\bm{p}'| \cos\theta \\
\end{split}
\end{equation*}
Energy conservation:
\begin{equation*}
    E_1 + E_2 = E_3 + E_4
\end{equation*}
Mandelstam variables:
\begin{equation*}
\begin{split}
    s &= (p_1 + p_2)^2 = m_1^2 + m_2^2 + 2 p_1 \cdot p_2 \\
      &= (p_3 + p_4)^2 = m_3^2 + m_4^2 + 2 p_3 \cdot p_4 \\
    t &= (p_1 - p_3)^2 = m_1^2 + m_3^2 - 2 p_1 \cdot p_3 \\
      &= (p_2 - p_4)^2 = m_2^2 + m_4^2 - 2 p_2 \cdot p_4 \\
    u &= (p_1 - p_4)^2 = m_1^2 + m_4^2 - 2 p_1 \cdot p_4 \\
      &= (p_2 - p_3)^2 = m_2^2 + m_3^2 - 2 p_2 \cdot p_3 \\
\end{split}
\end{equation*}

\subsubsection{s-channel}
Assume $m = m_1 = m_2$ and $M = m_3 = m_4$
\begin{equation*}
\begin{split}
    E_1 &= E_2 \\
    E_3 &= E_4 \\
    |\bm{p}'| &= |\bm{p}| \cos\theta \\
\end{split}
\end{equation*}
Dot products
\begin{equation*}
\begin{split}
    p_1 \cdot p_3 &= p_2 \cdot p_4 = \frac{m^2 + M^2 - t}{2} \\
        &= E E' - \bm{p} \cdot \bm{p}' = E^2 - \sqrt{(E^2-m^2)(E^2-M^2)} \cos\theta \\
    p_1 \cdot p_4 &= p_2 \cdot p_3 = \frac{m^2 + M^2 - u}{2} \\
        &= E E' + \bm{p} \cdot \bm{p}' = E^2 + \sqrt{(E^2-m^2)(E^2-M^2)} \cos\theta \\
    p_1 \cdot p_2 &= \frac{s - 2m^2}{2} \\
    p_3 \cdot p_4 &= \frac{s - 2M^2}{2} \\
    s &= 4E^2 \\
\end{split}
\end{equation*}

\subsubsection{t-channel}

\subsubsection{u-channel}

\section{Cross Sections and Decay Rates}

\subsection{Fermi's Golden Rule}
\textbf{Non-Relativistic Case:}
Consider the transition matrix element from $\ket{n}$ to $\ket{m}$ at first order in the Dyson series expansion:
\begin{equation*}
    T_{n \to m} = -i \bra{m} \int_{-\infty}^\infty H_I(t) \diff t \ket{n} = -i \bra{m} H_{int} \ket{n} 2\pi\delta(\omega)
\end{equation*}
The probability is
\begin{equation*}
\begin{split}
    P_{n \to m}
        &= |T_{n \to m}|^2 = |\bra{m} H_{int} \ket{n}|^2 (2\pi)^2 \delta(\omega)^2 = |\bra{m} H_{int} \ket{n}|^2 2\pi\delta(\omega) T \\
\end{split}
\end{equation*}
\textbf{Fermi's Golden Rule}:
\begin{equation*}
    W_{n \to m} = \lim_{T \to \infty} \frac{P_{n \to m}}{T} = |\bra{m} H_{int} \ket{n}|^2 2\pi\delta(\omega)
\end{equation*}

\textbf{Relativistic Case:}
We may express the transition matrix through the matrix element $\mathcal{M}$:
\begin{equation*}
\begin{split}
    T_{i \to f}
        &= -i \bra{f} \int_{-\infty}^\infty H_I(t) \ket{i} = -i \bra{f} \int V_I(x) \diff^4 x \ket{i} = -i \mathcal{M} (2\pi)^4 \delta^4(p_i - p_f) \\
    P_{i \to f}
        &= \frac{|\mathcal{M}|^2 (2\pi)^4 \delta^4(p_i - p_f) V T}{\braket{i|i} \braket{f|f}} = |\mathcal{M}|^2 (2\pi)^4 \delta^4(p_i - p_f) V T \prod_{i} \frac{1}{2E_i V} \prod_{f} \frac{1}{2E_f V} \\
\end{split}
\end{equation*}
We introduce the measure for the density of the final states:
\begin{equation*}
\begin{split}
    \diff W_{i \to f}
        &= |\mathcal{M}|^2 (2\pi)^4 \delta^4(p_i - p_f) V \prod_i \frac{1}{2E_i V} \prod_f \frac{1}{2E_f V} \frac{V \diff^3 p_f}{(2\pi)^3} \\
        &= |\mathcal{M}|^2 (2\pi)^4 \delta^4(p_i - p_f) V \prod_i \frac{1}{2E_i V} \prod_f \frac{\diff^3 p_f}{(2\pi)^3 2E_f} \\
\end{split}
\end{equation*}

\subsection{Cross Section Formulae}

\textbf{Definition}: Let $F$ be the incident flux and $W$ be the transition rate. The cross-section is 
defined as:
\begin{equation*}
    F \diff \sigma = \diff W
\end{equation*}
The flux is given by
\begin{equation*}
\begin{split}
    F
        &= \frac{N_1 N_2 |\bm{v}_1 - \bm{v}_2|}{V}
        = \frac{N_1 N_2}{V} \left| \frac{\bm{p}_1}{E_1} - \frac{\bm{p}_2}{E_2} \right| \\
\end{split}
\end{equation*}

$2 \to 2$ \textbf{Differential Cross Section in CoM Frame}:
\begin{equation*}
    \frac{\diff \sigma}{\diff \Omega} = \left( \frac{\hbar c}{8\pi} \right)^2 \frac{\mathcal{S} \overline{|\mathcal{M}|^2}}{(E_1 + E_2)^2} \frac{|\bm{p}_f|}{|\bm{p}_i|}
\end{equation*}

$2 \to N$ \textbf{Differential Cross Section in CoM Frame}:
\begin{equation*}
    \diff \sigma = \frac{\hbar^2 \mathcal{S}}{4 \sqrt{(p_1 \cdot p_2)^2 - (m_1 m_2)^2} } \overline{|\mathcal{M}|^2} \left[ \frac{c \diff^3 p_3}{(2\pi)^3 2E_3} \dots \frac{c \diff^3 p_N}{(2\pi)^3 2E_N} \right] (2\pi)^4 \delta(p_1 + p_2 - p_3 - ... - p_N)
\end{equation*}

\subsection{Decay Rate Formulae}

\textbf{Definition}: The decay rate is just the transition rate from a single-particle initial state at rest:
\begin{equation*}
\begin{split}
    \diff \Gamma
        &= \diff W_{1 \to f}
        = \frac{1}{2m} |\mathcal{M}|^2 (2\pi)^4 \delta^4(p_i - p_f) \prod_f \frac{\diff^3 p_f}{(2\pi)^3 2E_f} \\
\end{split}
\end{equation*}


\subsection{Interpretations of Cross Sections}
Consider a differential cross section $\diff \sigma / \diff a$ where $a$ is some parameter. Then the probability
the scattering occurs in a range $[a, a + \diff a]$ is
\begin{equation*}
    p(a) \diff a = \frac{1}{\sigma} \frac{\diff \sigma}{\diff a} \diff a
\end{equation*}
Let $N_i$ be the number of events measured by a detector in the $i$th bin $[a_i, a_i + \delta a]$ for some
detector resolution $\delta a$. The total number of events collected is:
\begin{equation*}
    N_{tot} = \sum_i N_i 
\end{equation*}
Then the scattering probability is measured by
\begin{equation*}
    \frac{N_i}{N_{tot}} = \frac{1}{\sigma} \frac{\diff \sigma(a_i)}{\diff a} 
\end{equation*}


\section{Perturbation Theory}

\subsection{Interaction / Dirac Picture}
We'll be doing this with real scalar fields $\phi$. We'll omit the subscript for full theory, use
$F$ for free theory, and $I$ for interaction picture. Furthremore, the interaction does NOT involve
any derivatives of the field. 

\textbf{Interaction Lagrangian and Hamiltonian}
\begin{equation*}
\begin{split}
    \mathcal{L} &= \mathcal{L}_0 + \mathcal{L}_{int} \\
    H &= H_0 + H_{int} = H_0 + \int \diff^3 x \, \mathcal{L}_{int} \\
\end{split}
\end{equation*}
we wish to rewrite $\phi$ and the vacuum $\ket{\Omega}$ in the interacting theory in terms of $\phi_F$ and
$\ket{0}$ in the free theory.

\textbf{Interacting Field}: For $\mathcal{L}_{int}$ small perturbations, the time evolution of
$\phi$ by $H_0$ is the dominant term:
\begin{equation*}
    \phi_I(t, \bm{x}) = e^{iH_0 (t-t_0)} \phi_F(t_0, \bm{x}) e^{-iH_0(t-t_0)} = \phi_F(t, \bm{x})
\end{equation*}
The Heisenberg picture of the full theory is related by:
\begin{equation*}
\begin{split}
    \phi(t, \bm{x})
        % &= e^{iH(t-t_0)} \phi_F(t_0, \bm{x}) e^{-iH(t-t_0)} \\
        % &= e^{iH(t-t_0)} e^{-iH_0(t-t_0)} \phi_I(t,\bm{x}) e^{iH_0(t-t_0)} e^{-iH(t-t_0)} \\
        &= U_I^\dagger(t, t_0) \phi_I(t, \bm{x}) U_I(t,t_0) \\
    U_I(t,t_0) &= e^{iH_0(t-t_0)} e^{-iH(t-t_0)} \\
\end{split}
\end{equation*}

\textbf{Interaction Picture Time Evolution}: The operator $U_I(t,t_0)$ is the time evolution operator
in the interaction picture. It solves the Schrodinger equation with initial condition $U_I(t_0, t_0) = 1$:
\begin{equation*}
\begin{split}
    i \frac{\diff }{\diff t} U(t,t_0) &= H_I(t) U(t, t_0) \\
    H_I(t) &= e^{iH_0(t-t_0)} H_{int} e^{-iH_0(t-t_0)} \\
\end{split}
\end{equation*}
Since the interaction doesn't involve derivatives of the field, $H_I$ is the same as $H_{int}$ but with
$\phi \to \phi_I$. Expanding $U_I$ in a power series of $H_I$ and using the time ordering operator $\mathcal{T}$
to convert the time integrals to be over $[t_0, t]$ gives the \textbf{Dysons Series}:
\begin{equation*}
    U(t,t_0) = \mathcal{T} \left\{ \exp\left[ -i \int_{t_0}^{t} \diff t' \, H_I \right]  \right\} 
\end{equation*}

\textbf{Interacting Picture Vacuum}: Let $\ket{n}$ be energy eigenstates of $H$ with $\ket{n=0} = \ket{\Omega}$.
Evolution of the free vacuum in the full theory:
\begin{equation*}
\begin{split}
    e^{-iHT} \ket{0}
        &= e^{-iE_0 T} \ket{\Omega} \braket{\Omega|0} + \sum_{n \neq 0} e^{-iE_n T} \ket{n} \braket{n|0} \\
    \ket{\Omega}
        % &= \lim_{T \to \infty(1-i\epsilon)} \frac{e^{-i(H-E_0)T}}{\braket{\Omega|0}} \ket{0} \\
        &= \lim_{T \to \infty(1-i\epsilon)} \frac{e^{iE_0(t_0 - (-T))}}{\braket{\Omega|0}} U(t_0, -T) \ket{0} \\
\end{split}
\end{equation*}

\textbf{2-pt Correlator}:
For $x^0 > y^0 > t^0$ reference time, the 2-point correlator becomes
\begin{equation*}
    \frac{\bra{\Omega}\phi(x) \phi(y) \ket{\Omega}}{\braket{\Omega|\Omega} }
    = \lim_{T \to \infty(1-i\epsilon)} \frac{\bra{0}U(T,x^0) \phi_I(x) U(x^0, y^0) \phi_I(y) U(y^0, -T)\ket{0}}{\bra{0}U(T,-T)\ket{0}}
\end{equation*}
With the normalization $\braket{\Omega|\Omega} = 1$ and for arbitrary times $x^0$ and $y^0$:
\begin{equation*}
    \bra{\Omega}\mathcal{T} \phi(x) \phi(y) \ket{\Omega}
    =
    \lim_{T \to \infty(1-i\epsilon)} \frac{\bra{0}\mathcal{T}\left\{\phi_I(x) \phi_I(y) \exp\left[ -i\int_{-T}^T H_I(t) \diff t \right] \right\}\ket{0}}{\bra{0} \mathcal{T} \left\{ \exp \left[ -i\int_{-T}^T H_I(t) \diff t \right] \right\} \ket{0} }
\end{equation*}

\subsection{Wick's Theorem}

\textbf{Creation / Annihilation Decomposition}:
\begin{equation*}
\begin{split}
    \phi &= \phi^+ + \phi^- \\
    \phi^+ \ket{0} = 0 \qquad & \qquad \bra{0} \phi^- = 0 \\
\end{split}
\end{equation*}

\textbf{Normal Ordering}: All $\phi^+$ operators are to the right of all of the $\phi^-$ operators:
\begin{equation*}
\begin{split}
    \phi(x)\phi(y)
        &= \phi^+(x) \phi^+(y) + \phi^-(x) \phi^+(y) + \phi^+(x) \phi^-(y) + \phi^-(x) \phi^-(y) \\
        &= \phi^+(x) \phi^+(y) + \phi^-(x) \phi^+(y) + \phi^-(y) \phi^+(x) + \phi^-(x) \phi^-(y) + [\phi(x), \phi^(y)] \\
        &= \, \normord{\phi(x)\phi(y)} + [\phi(x), \phi(y)] \\
    \mathcal{T} \left\{ \phi(x) \phi(y) \right\} &= \, \normord{\phi(x)\phi(y)} + D_F(x - y) \\
\end{split}
\end{equation*}
where $D_F(x - y)$ is the Feynamn propagator from $x$ to $y$. The propagator is also known as a contraction
between $\phi(x)$ and $\phi(y)$.

\textbf{Wick's Theorem}:
\begin{equation*}
    \mathcal{T} \left\{ \phi(x_1) \dots \phi(x_N) \right\} = \, \normord{\phi(x_1) \dots \phi(x_N) + \text{ all contractions }}
\end{equation*}
Only fully contracted terms appear in correlation functions
\begin{equation*}
    \bra{0} \mathcal{T} \left\{ \phi(x_1) \dots \phi(x_N) \right\} \ket{0} = \bra{0} \text{ fully contracted terms } \ket{0}
\end{equation*}

\subsection{S-Matrix}
\begin{defn}[S-Matrix]
    Let $\ket{i}$ and $\ket{f}$ be initial and final states respectively. The \textbf{S-matrix} is the
    unitary time-evolution operator between the initial and final states:
    \begin{equation*}
        \bra{f} S \ket{i} = \lim_{t \to \infty} \bra{f} U(t, -t) \ket{i}
    \end{equation*}
\end{defn}
The initial state may be written as some creation operators at $t = -\infty$ and similarly for the
final state at $t = \infty$. We may calculate S-matrix terms by perturbative expansions of the Dysons series:
\begin{equation*}
\begin{split}
    \bra{f} S \ket{i}
        &= \sum_{n = 0}^\infty \frac{1}{n!} \bra{f} \mathcal{T} \left[ \int H_I \diff^4 x \right]^n \ket{i} \\
        &= \sum_{n = 0}^\infty \frac{1}{n!} \bra{0} \phi^\dagger_{f} \mathcal{T} \left[ \int H_I \diff^4 x \right]^n \phi_i \ket{0}
        = \sum_{n = 0}^\infty \frac{1}{n!} \bra{0} \mathcal{T} \phi^\dagger_{f} \phi_i \left[ \int H_I \diff^4 x \right]^n \ket{0} \\
\end{split}
\end{equation*}
Since $H_I$ is an operator made up from many field operators $\phi$, the final term may be computed via
Wick's theorem.

\subsection{External Legs}
\textbf{Wick Contractions With External Fields}: Consider the matrix element
\begin{equation*}
\begin{split}
    \bra{\bm{p}_1, \bm{p}_2} \mathcal{T}\left\{ \phi(x_1) \dots \phi(x_n) \right\} &\ket{\bm{k}_1, \bm{k}_2}
        = \bra{\bm{p}_1, \bm{p}_2} :\phi(x_1) \dots \phi(x_n) + \text{all contractions}: \ket{\bm{k}_1, \bm{k}_2} \\
        &= \bra{0} \phi^+(p_1) \phi^+(p_2) :\phi(x_1) \dots \phi(x_n) + \text{all contractions}: \phi^-(k_1) \phi^-(k_2) \ket{0} \\
\end{split}
\end{equation*}
Note how the fields corresponding to the initial and final particles are not normal ordered, and are also
not inside the normal ordering operator. We can define contractions of fields into the states which
define the external leg Feynman rules. 

\section{Non-Relativistic Scalar Field}

\textbf{Lagrangian}:
\begin{equation*}
    \mathcal{L} = \psi^* i \partial_t \psi - \frac{1}{2M} (\nabla\psi)^* \cdot (\nabla\psi)
\end{equation*}

\textbf{Field Solution}:
\begin{equation*}
    \psi(\bm{x}) = \int \frac{\diff^3 k}{(2\pi)^3} a_k e^{i \bm{k} \cdot \bm{x}}  \qquad \qquad
    \psi^\dagger(\bm{x}) = \int \frac{\diff^3 k}{(2\pi)^3} a^\dagger_k e^{-i \bm{k} \cdot \bm{x}} 
\end{equation*}

\section{Quantum Chromodynamics (QCD)}

\subsection{Form Factors}
\textbf{Static Form Factor}: For a static charge distribution $\rho(\bm{x})$, the form factor
$F(\bm{q})$ is its Fourier transform:
\begin{equation*}
    F(\bm{q}) = \int \rho(\bm{x}) e^{i \bm{q} \cdot \bm{x}} \diff^3 x
\end{equation*}
The cross section for scattering off of the distribution is related to scattering off a point charge
inside the distribution
\begin{equation*}
    \frac{\diff \sigma}{\diff \Omega} = \left( \frac{\diff \sigma}{\diff \Omega}  \right)_{\text{point}} |F(\bm{q})|^2 
\end{equation*}

\textbf{Proton Form Factors}: For a charge distribution in motion, we may instead rewrite their interaction
vertex as a sum of QED vertices at all orders $i\gamma^\mu \to i\Gamma^\mu(q)$, where
\begin{equation*}
    i\Gamma^\mu(q) = F_1 (q^2) + \frac{\kappa}{2M} F_2(q^2) i \sigma^{\mu\nu} q_\nu
\end{equation*}
and $\kappa$ is the anomalous magnetic moment
\begin{equation*}
    \kappa = \frac{g-2}{2}
\end{equation*}

\textbf{Electric and Magnetic Form Factors}: Lab frame elastic electron-proton scattering (Rosenbluth Formula):
\begin{equation*}
\begin{split}
     \left. \frac{\diff \sigma}{\diff \Omega} \right|_{lab}
        &= \frac{\alpha^2}{4E^2 \sin^4(\theta/2)} \frac{E'}{E} \left[ \left( F_1^2 - \frac{\kappa^2 q^2}{4M^2} F_2^2 \right) \cos^2 \frac{\theta}{2} - \frac{q^2}{2M^2} (F_1 + \kappa F_2)^2 \sin^2 \frac{\theta}{2} \right] \\
        &= \frac{\alpha^2}{4E^2 \sin^4(\theta/2)} \frac{E'}{E} \left[ \frac{G_E^2 + \tau G^2_M}{1 + \tau} \cos^2 \frac{\theta}{2} + 2\tau G^2_M \sin^2 \frac{\theta}{2} \right] \\
     \tau &= -\frac{q^2}{4M^2} \\
\end{split}
\end{equation*}
where $G_E$ and $G_M$ are linear combinations of $F_1$ and $F_2$ that make the interference terms vanish:
\begin{equation*}
    G_E = F_1 + \frac{\kappa q^2}{4M^2} F_2 \qquad \qquad G_M = F_1 + \kappa F_2
\end{equation*}

\subsection{Deep Inelastic Scattering}
\textbf{Hadronic Tensor}:
\begin{equation*}
    W^{\mu\nu} = -W_1 g^{\mu\nu} + \frac{W_2}{M^2} p^\mu p^\nu + \frac{W_4}{M^2} q^\mu q^\nu + \frac{W_5}{M^2} (p^\mu q^\nu + q^\mu p^\nu)
\end{equation*}
Note that $W$ is symmetric. Current conservation implies $q_\mu W^{\mu\nu} = 0$, and
\begin{equation*}
    W_5 = - \frac{p \cdot q}{q^2} W_2 \qquad \qquad W_4 = \left( \frac{p \cdot q}{q^2} \right)^2 W_2 + \frac{M^2}{q^2} W_1
\end{equation*}
and thus in terms of $W_1$ and $W_2$:
\begin{equation*}
    W^{\mu\nu} = W_1 \left( -g^{\mu\nu} + \frac{q^\mu q^\nu}{q^2} \right) + W_2 \frac{1}{M^2} \left( p^\mu - \frac{p \cdot q}{q^2} q^\mu \right) \left( p^\nu - \frac{p \cdot q}{q^2} q^\nu \right) 
\end{equation*}

\section{Weak Interactions}

\subsection{4-pt Approximation}
Fermi's original guess for weak interactions was a 4-point interaction involving generalized currents.
Compare $pe \to pe$ via EM interactions versus $pe \to n \nu_e$ via weak interaction:
\begin{equation*}
\begin{split}
    \mathcal{M}_{em} &= \frac{-e^2}{q^2} \left( \overline{u}_e \gamma^\mu u_e \right) \left( \overline{u}_p \gamma_\mu u_p \right)_\mu \\
    \mathcal{M}_{weak} &= G \left( \overline{u}_\nu \gamma^\mu u_e \right) \left( \overline{u}_n \gamma_\mu u_p \right) \\
\end{split}
\end{equation*}
where $G$ is Fermi's constant.

\end{document}
