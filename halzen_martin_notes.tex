\documentclass[12pt]{article}
\usepackage[letterpaper, portrait, margin=0.5in]{geometry}
\usepackage[utf8]{inputenc}

\usepackage{amsmath}
\usepackage{amsthm}
\usepackage{amssymb}
\usepackage{braket}
\usepackage{bm}

\setlength{\parskip}{1em}

\newcommand{\diff}{\mathop{}\!\mathrm{d}}
\newcommand{\R}{\mathbb{R}}
\newcommand{\C}{\mathbb{C}}

\theoremstyle{definition}
\newtheorem{defn}{Defn}[section]
\newtheorem{thm}{Theorem}[section]
\newtheorem{corl}{Corollary}[section]

\newenvironment{problem}[1]
    {\noindent
    \textbf{#1})
    }
    {
    \vspace{12pt}
    }

\begin{document}

\section{Free Particle Wavefunctions}

\subsection{Non-Relativistic Particles}
Schrodinger equation
\begin{equation*}
    0 = i \partial_t \psi - H \psi = \left( i \partial_t + \frac{\nabla^2}{2m} \right) \psi
\end{equation*}
Solution, where $E = \bm{p}^2 / 2m$:
\begin{equation*}
    \psi_p = N e^{-iEt + i \bm{p} \cdot \bm{x}}
\end{equation*}
Density and current
\begin{alignat*}{3}
    & \qquad && \rho && \bm{j} \\
    & \text{Equation} \qquad && |\psi|^2 \qquad && \frac{1}{2mi} \left( \psi^* \nabla \psi - \psi \nabla \psi^* \right) \\
    & \text{Free} \qquad && |N|^2 \qquad && \frac{\bm{p}}{m} |N|^2 \\
\end{alignat*}

\subsection{Scalar Field Equation}
Klein-Gordon Equation
\begin{equation*}
    (-\partial_t^2 + \nabla^2) \phi = m^2 \phi
\end{equation*}
Solution, where $p^0 = E = \sqrt{\bm{p}^2 + m^2}$:
\begin{equation*}
    \phi = N e^{-iEt + i\bm{p} \cdot \bm{x}} = N e^{- i p \cdot x}
\end{equation*}
Density and current
\begin{alignat*}{4}
    & \qquad && \rho && \bm{j} && j^\mu \\
    & \text{Equation} \qquad && i \left( \phi^* \partial_t \phi - \phi \partial_t \phi^* \right) \qquad && -i\left( \phi^* \nabla \phi - \phi \nabla \phi^* \right) \qquad && i\left( \phi^* \partial^\mu \phi - \phi \partial^\mu \phi^* \right) \\
    & \text{Free} \qquad && 2E|N|^2 && 2\bm{p}|N|^2 \qquad && 2p^\mu |N|^2 \\
\end{alignat*}
Antiparticles (Feynman-Stuckelberg Interpretation)
\begin{equation*}
\begin{split}
    j^\mu(\phi,-p) &= 2p^\mu |N|^2 = j^\mu(\phi^*,p) \\
    \phi^* &= N e^{-i (-p) \cdot x} = N e^{i p \cdot x} \\
\end{split}
\end{equation*}
Identify antiparticles $\phi^*$ as having opposite charge in the current. Antiparticles have positive energy
and can be used instead of equivalent negative energy solutions.

\subsection{Electrodynamics}
Minimal coupling prescription
\begin{equation*}
    i \partial^\mu \to i \partial^\mu + e A^\mu 
\end{equation*}

\subsection{Scalar QED}
Perturbative potential
\begin{equation*}
    V = -ie(\partial_\mu A^\mu + A^\mu \partial_\mu) - e^2 A^2
\end{equation*}
Electromagnetic current between states $i \to f$
\begin{equation*}
    j^{fi}_\mu = -ie(\phi^*_f \partial_\mu \phi_i - \phi_i \partial_\mu \phi^*_f) = -e N_i N_f (p_i + p_f)_\mu e^{i(p_f - p_i) \cdot x} 
\end{equation*}

Photon equation of motion and solution (from Maxwell's equations)
\begin{equation*}
\begin{split}
    (-\partial_t^2 + \nabla^2) A^\mu &= j^\mu  \\
    A^\mu &= -\frac{1}{q^2} j^\mu \\
        &= \frac{-ig^{\mu\nu}}{q^2} j_\nu \\
\end{split}
\end{equation*}
Photon propagator
\begin{equation*}
    \frac{-i g^{\mu\nu}}{q^2}
\end{equation*}

\section{Perturbation Theory}

State transistion amplitude $i \to f$:
\begin{equation*}
\begin{split}
    V_{fi}(t) &= \int \diff^3 x \, \phi_f^*(\bm{x},t) V(\bm{x},t) \phi_i(\bm{x},t) \\
    T_{fi} &= -i \int \diff^4 x \phi^*_f(x) V(x) \phi_i(x) \\
        &= -i \int_{-\infty}^\infty V_{fi}(t) \diff t \\
\end{split}
\end{equation*}

Fermi's Golden Rule (assuming $V$ is time-independent)
\begin{equation*}
\begin{split}
    W_{fi}
        &= \int \rho(E_f) \diff E_f \, \lim_{T \to \infty} \frac{|T_{fi}|^2}{T} \\
        &= 2\pi |V_{fi}|^2 \int \rho(E_f) \diff E_f \, \delta(E_f - E_i) \lim_{T \to \infty} \frac{1}{T} \int_{-T/2}^{T/2} \diff t \\
        &= 2\pi |V_{fi}|^2 \rho(E_i) \\
\end{split}
\end{equation*}
 
State transition amplitude for $2 \to 2$ scattering (scalar QED):
\begin{equation*}
\begin{split}
    T_{fi} &= -i \int j^{(1)}_\mu A^\mu \diff^4 x \\
        &= -i \int j^{(1)}_\mu \frac{-i g^{\mu\nu}}{q^2} j^{(2)}_\nu \diff^4 x \\
        &= N_A N_B N_C N_D (2\pi)^4 \delta(p_A + p_B - p_C - p_D) (-i\mathcal{M}) \\
    -i\mathcal{M} &= ie(p_A + p_C)^\mu \left( -i \frac{g_{\mu\nu}}{q^2} \right) ie(p_B + p_D)^\nu \\
\end{split}
\end{equation*}

\section{Cross-Section}

The transition matrix element is
\begin{equation*}
    T_{n \to m}
        = -i \bra{m} \int_{-\infty}^\infty H_I(t) \diff t \ket{n} = -i \bra{m} H_{int} \ket{n} 2\pi\delta(\omega) 
\end{equation*}
The probability is then
\begin{equation*}
    P_{n \to m}
        = |T_{n \to m}|^2 = |\bra{m} H_{int} \ket{n}|^2 (2\pi)^2 \delta(\omega)^2
\end{equation*}
To compute the squared delta function, rewrite it as a Fourier transform over some finite time $T$ and use
the first delta function to set the second:
\begin{equation*}
\begin{split}
    (2\pi)^2 \delta(\omega)^2
        &= 2\pi\delta(\omega) \lim_{T \to \infty} \int_{-T/2}^{T/2} e^{-i\omega t} \diff t \\
        &= 2\pi\delta(\omega) \lim_{T \to \infty} \int_{-T/2}^{T/2} \diff t \\
        &= 2\pi\delta(\omega) \lim_{T \to \infty} T \\
\end{split}
\end{equation*}
This is obviously divergent, so instead we take the probability rate
\begin{equation*}
\begin{split}
    W_{n \to m}
        &= \lim_{T \to \infty} \frac{P_{n \to m}}{T} = |\bra{m} H_{int} \ket{n}|^2 2\pi\delta(\omega)
\end{split}
\end{equation*}


\end{document}
