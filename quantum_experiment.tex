\documentclass{article}
\usepackage[utf8]{inputenc}
\usepackage[a4paper, portrait, margin=0.75in]{geometry}

\usepackage{amsmath}
\usepackage{amsthm}
\usepackage{amssymb}

\usepackage{tikz}

\begin{document}

\tableofcontents
\pagebreak

\section{Photoelectric Effect}

\subsection{Experiment}
A cathode is held at a potential $+V$, while a metal anode is connected to ground.
Monochromatic light is shined on the anode, causing it to emit electrons. When the
electrons hit the cathode, the circuit is completed, causing a discharge current 
$i_{\text{discharge}}$ to flow. Notably, the discharge current occurs whenever
the kinetic energy of the electrons exceeds the potential barrier of the voltage:
\begin{equation*}
	E_{\text{kin}} > e \cdot V
\end{equation*}

For a fixed wavelength of light $\nu$ shown at intensity $I$, the discharge current
is 0 up until some voltage $V_0$ called the \emph{stopping voltage}, and then increased
until a saturation point. The saturation current is proportional to the intensity of
the light.

The stopping voltage $V_0$ does not depend upon the intensity, but does depend on
the frequency $\nu$, for which it is 0 up until some $\nu_0$ then linearly increases
with $\nu$. Furthermore, once the process begins, electrons are ejected almost
immediately (roughly $10^{-9}$ seconds). 

\subsection{Classical Predictions}
Imagine an electron in some potential well in the metal, with some energy difference
(called the work function $w$) required to escape. Assuming the electron lives in
a sphere of Bohr radius $a_B$, then we can estimate the escape time as:
\begin{equation*}
	I = \frac{E}{St} \quad \implies \quad t = \frac{w}{a_B^2 I} \approx 10^{-2} s
\end{equation*}
which is seven orders of magnitude difference. The kinetic energy of the electron
may be determined by approximating the electron has being driven by the electric
field wave:
\begin{equation*}
	m \ddot{x} = e \mathcal{E} \cos \omega t
\end{equation*}
So the kinetic energy is roughly:
\begin{equation*}
	\frac{1}{2} m \dot{x}^2 \propto \mathcal{E}^2 \frac{1}{\omega^2} = eV_0
\end{equation*}
As the frequency of the light increases, then the stopping voltage should be smoothly
decreasing! The electrons have greater kinetic energy, and therefore should be able
to readily cross the vacuum to the cathode for higher frequency light.

\subsection{Quantum Theory}
Here we only need to quantize electrons in the metal. That is, they accept energy
from light proportional to its frequency:
\begin{equation*}
	E_{\text{kin}} = h \nu - w
\end{equation*}

The stopping voltage is then given by
\begin{equation*}
	V_0 = \max\left\{ 0, \frac{1}{e}(h\nu - w) \right\}
\end{equation*}

and the stopping frequency by:
\begin{equation*}
	\nu_0 = \frac{1}{h} w
\end{equation*}

\section{Double-Slit Experiment}
Consider particles moving perpendicular to a plane with two slits. Classically, we
expect the opposite wall to have two Gaussian peaks centered on the slits. However,
we instead see that the diffraction matches the pattern of a wave.

Let $\omega$ be the frequency of the incident plane wave, and $r_1,r_2$ the distance
from the measurement point to the first and second slit respectively. We can write
the amplitude as
\begin{equation*}
\begin{split}
	A
		&= A_0 \cos(kr_1 - \omega t) + A_0 \cos(kr_2 - \omega t) \\
		&= 2A_0 \cos\left(k\frac{r_1 + r_2}{2} - \omega t\right)\cos\left(k\frac{r_1-r_2}{2}\right) \\
\end{split}
\end{equation*}
The intensity is:
\begin{equation*}
	I = |A|^2 = A_0^2 \cos^2\left(k\frac{r_1+r_2}{2}-\omega t\right) \cos^2\left(k\frac{r_1-r_2}{2}\right)
\end{equation*}
when averaged over a period, $\cos^2(\dots - \omega t) = 1/2$. The maxima are then
at
\begin{equation*}
	\frac{2\pi}{\lambda} \frac{|r_1-r_2|}{2} = n\pi
\end{equation*}
where $n \in \mathbb{Z}$. 

\section{Davisson-Germer}

\subsection{Experiment}
An electron gun is aimed at a perpendicular sheet of single metal crystal. The experiment
measures the number of electrons deflected off of the crystal at a given angle.

Classically, we expect the count to smoothly decrease as the angle of deflection
increases. However, at some angle $\theta_0(V)$ (as a function of voltage)
there's a spike in counts.

\subsection{Quantum Theory}
Diffraction of electrons as though they were waves, which again provides evidence
for wave-particle duality. The wavelength of the electron is given by:
\begin{equation*}
	\lambda_{dB} = \frac{h}{p}
\end{equation*}
where $p$ is its momentum. At a high enough voltage, the ejected electron has a
wavelength short enough to probe the lattice spacing of the crystal.

The results are described by Bragg's law. Consider the first two layers of the
crystal. An incident plane wave bounces off the top layer, and reflects off of it.
The same plane wave bounces off the second layer directly below the first reflection,
picking up a phase as it leaves the crystal, which is constructive precisely when
\begin{equation*}
	2 d \sin \theta = n \lambda_{dB}
\end{equation*}
for some $n \in \mathbb{Z}$, incident angle $\theta$, and lattice spacing $d$.

\section{Stern-Gerlach}

\end{document}
