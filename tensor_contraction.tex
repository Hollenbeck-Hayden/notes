\documentclass[12pt]{article}
\usepackage[letterpaper, portrait, margin=0.5in]{geometry}
\usepackage[utf8]{inputenc}

\usepackage{amsmath}
\usepackage{amsthm}
\usepackage{amssymb}
\usepackage{braket}
\usepackage{bm}

\setlength{\parskip}{1em}

\newcommand{\diff}{\mathop{}\!\mathrm{d}}
\newcommand{\R}{\mathbb{R}}
\newcommand{\C}{\mathbb{C}}

\theoremstyle{definition}
\newtheorem{defn}{Defn}[section]
\newtheorem{thm}{Theorem}[section]
\newtheorem{cor}{Corollary}[section]
\newtheorem{prop}{Proposition}[section]

\newenvironment{problem}[1]
    {\noindent
    \textbf{#1})
    }
    {
    \vspace{12pt}
    }

\newcommand{\sym}{\text{Sym }}
\newcommand{\antisym}{\text{AntiSym }}

\begin{document}

\section{Symmetrization of Rank-2 Tensors}

\begin{defn}[Symmetric and Antisymmetric Tensor]
    A rank-2 tensor $S^{\mu\nu}$ is \textbf{symmetric} if it's the same under exchange of its indices
    $\mu$ and $\nu$:
    \begin{equation*}
        S^{\mu\nu} = S^{\nu\mu}
    \end{equation*}
    Examples: stress-energy tensor $\mathcal{T}^{\mu\nu}$; Kronecker-Delta $\delta^{\mu\nu}$; Lorentz metric
    $\eta^{\mu\nu}$.
    
    A rank-2 tensor $A^{\mu\nu}$ is \textbf{antisymmetric} if it's the opposite under exchange of its indices:
    \begin{equation*}
        A^{\mu\nu} = -A^{\nu\mu}
    \end{equation*}
    Examples: electromagnetic field tensor $F^{\mu\nu}$
\end{defn}

\begin{defn}[Symmetrization and Antisymmetrization]
    Given an arbitrary rank-2 tensor $T^{\mu\nu}$, we may define its \textbf{symmetrization} as
    \begin{equation*}
        \sym T^{\mu\nu} = T^{[\mu\nu]} = \frac{T^{\mu\nu} + T^{\nu\mu}}{2}
    \end{equation*}
    and its \textbf{antisymmetrization} as
    \begin{equation*}
        \antisym T^{\mu\nu} = T^{\{\mu\nu\}} = \frac{T^{\mu\nu} - T^{\nu\mu}}{2}
    \end{equation*}
\end{defn}

For the following corollaries, let $T^{\mu\nu}$ be an arbitrary tensor, $S^{\mu\nu}$ be an arbitrary
symmetric tensor, and $A^{\mu\nu}$ be an arbitrary antisymmetric tensor.

\begin{prop}[Symmetric/Antisymmetric Decomposition]
    Symmetrization and antisymmetrization form a complete, disjoint set of projection operators of
    the set of rank-2 tensors onto the sets of symmetric and antisymmetric rank-2 tensors respectively.
    That is, for any given tensor $T^{\mu\nu}$, we may uniquely write it terms of symmetric and antisymmetric
    components:
    \begin{equation*}
        T^{\mu\nu} = T^{[\mu\nu]} + T^{\{\mu\nu\}}
    \end{equation*}
\end{prop}

\textbf{Proof:}
We'll first show that $\sym$ and $\antisym$ map tensors to symmetric and antisymmetric tensors respectively:
\begin{equation*}
\begin{split}
    \sym T^{\nu\mu} &= \frac{T^{\nu\mu} + T^{\mu\nu}}{2} = \sym T^{\mu\nu} \\
    \antisym T^{\nu\mu} &= \frac{T^{\nu\mu} - T^{\mu\nu}}{2} = - \antisym T^{\mu\nu} \\
\end{split}
\end{equation*}
Now we'll show that $\sym$ and $\antisym$ are projection operators:
\begin{equation*}
\begin{split}
    \sym \sym T^{\mu\nu} &= \sym \frac{T^{\mu\nu} + T^{\nu\mu}}{2} = \frac{1}{2} \left( \sym T^{\mu\nu} + \sym T^{\nu\mu} \right) \\
        &= \frac{1}{2} \left( \sym T^{\mu\nu} + \sym T^{\mu\nu} \right) = \sym T^{\mu\nu} \\
    \antisym \antisym T^{\mu\nu} &= \antisym \frac{T^{\mu\nu} - T^{\nu\mu}}{2} = \frac{1}{2} \left( \antisym T^{\mu\nu} - \antisym T^{\nu\mu} \right) \\
        &= \frac{1}{2} \left( \antisym T^{\mu\nu} + \antisym T^{\mu\nu} \right) = \antisym T^{\mu\nu} \\
\end{split}
\end{equation*}
And that they're disjoint projections:
\begin{equation*}
\begin{split}
    \sym \antisym T^{\mu\nu} = \sym \frac{T^{\mu\nu} - T^{\nu\mu}}{2} = \frac{T^{\mu\nu} + T^{\nu\mu} - T^{\nu\mu} - T^{\mu\nu}}{4} = 0 \\
    \antisym \sym T^{\mu\nu} = \antisym \frac{T^{\mu\nu} + T^{\nu\mu}}{2} = \frac{T^{\mu\nu} - T^{\nu\mu} + T^{\nu\mu} - T^{\mu\nu}}{4} = 0 \\
\end{split}
\end{equation*}
And finally, they're complete:
\begin{equation*}
\begin{split}
    (\sym + \antisym) T^{\mu\nu}
        &= \frac{T^{\mu\nu} + T^{\nu\mu}}{2} + \frac{T^{\mu\nu} - T^{\nu\mu}}{2} = T^{\mu\nu} \\
\end{split}
\end{equation*}

This should match your intuition for these words: there's no difference between a symmetric tensor and a
symmetrized tensor, and antisymmetric tensors have no symmetric component (and mutatus mutandi for 
antisymmetric tensors).

For some more examples, here's rank-2 tensors made out of rank-1 tensors (vectors):
\begin{equation*}
\begin{split}
    \sym p^\mu q^\nu &= \frac{p^\mu q^\nu + p^\nu q^\mu}{2} \\
    \antisym p^\mu q^\nu &= \frac{p^\mu q^\nu - p^\nu q^\mu}{2} \\
    \sym p^\mu p^\nu &= p^\mu p^\nu \\
    \antisym p^\mu p^\nu &= 0 \\
\end{split}
\end{equation*}
Please not that multiplying by a scalar doesn't change whether a tensor is symmetric or antisymmetric
or neither, so $p^\mu q^\nu + p^\nu q^\mu$ is still and symmetric tensor, it's just twice the symmetrization
of $p^\mu q^\nu$. 

\begin{prop}[Contraction Symmetric and Antisymmetric Tensors]
    The full contraction of a symmetric tensor and antisymmetric tensor is zero.
\end{prop}

\textbf{Proof:}
\begin{equation*}
\begin{split}
    (\sym T^{\mu\nu}) (\antisym T_{\mu\nu})
        &= \frac{1}{4} \left( T^{\mu\nu} + T^{\nu\mu} \right) \left( T_{\mu\nu} - T_{\nu\mu} \right) \\
        &= \frac{1}{4} \left( T^{\mu\nu} T_{\mu\nu} + T^{\nu\mu} T_{\mu\nu} - T^{\mu\nu} T_{\nu\mu} - T^{\nu\mu} T_{\nu\mu} \right) \\
        &= \frac{1}{4} \left( T^{\mu\nu} T_{\mu\nu} + T^{\nu\mu} T_{\mu\nu} - T^{\nu\mu} T_{\mu\nu} - T^{\mu\nu} T_{\mu\nu} \right) \\
        &= 0 \\
\end{split}
\end{equation*}

\begin{prop}[Contraction of Arbitrary Tensors]
    For two rank-2 tensors $T^{\mu\nu}$ and $U^{\mu\nu}$, their contraction is just the sum of the
    contractions of their (anti)symmetrized components:
    \begin{equation*}
        T^{\mu\nu} U_{\mu\nu} = (\sym T^{\mu\nu})(\sym U_{\mu\nu}) + (\antisym T^{\mu\nu})(\antisym U_{\mu\nu})
    \end{equation*}
\end{prop}
\textbf{Proof:} Follows immediately from expanding the tensors into their symmetric and antisymmetric
components, multiplying out the terms, and cancelling the cross terms by the previous propositions.

\begin{prop}[Contractions of Products of Vectors]
    We often come across contractions of sets of symmetric or antisymmetric vectors. Here's some useful
    identities.
    \begin{equation*}
    \begin{split}
        (p^\mu q^\nu + p^\nu q^\mu) (a_\mu b_\nu + a_\nu b_\mu) 
            &= 4 (\sym p^\mu q^\nu) (\sym a_\mu b_\nu) \\
            &= \left[ (p \cdot a) (q \cdot b) + (p \cdot b) (q \cdot a) + (p \cdot b) (q \cdot a) + (p \cdot a) (q \cdot b) \right]\\
            &= 2 \left[ (p \cdot a)(q \cdot b) + (p \cdot b)(q \cdot a) \right] 
    \end{split}
    \end{equation*}
    \begin{equation*}
    \begin{split}
        \left( p^\mu q^\nu - p^\nu q^\mu \right) \left( a_\mu b_\nu - a_\nu b_\mu \right)
            &= 4 (\antisym p^\mu q^\nu) (\antisym a_\mu b_\nu) \\
            &= \left[ (p \cdot a)(q \cdot b) - (p \cdot b)(q \cdot a) - (p \cdot b)(q \cdot a) + (p \cdot a)(q \cdot b) \right] \\
            &= 2 \left[ (p \cdot a)(q \cdot b) - (p \cdot b)(q \cdot a) \right] \\
    \end{split}
    \end{equation*}
    \begin{equation*}
    \begin{split}
        (p^\mu q^\nu + p^\nu q^\mu) (p_\mu q_\nu + p_\nu q_\mu)
            &= 4 (\sym p^\mu q^\nu) (\sym p_\mu q_\nu) \\
            &= 2 \left[ p^2 q^2 + (p \cdot q)^2 \right] \\
    \end{split}
    \end{equation*}
    Note that because of symmetry, the only terms we can make is $(p \cdot a)(q \cdot b)$ and $(p \cdot b)(q \cdot a)$
    up to signs and constant factors.
\end{prop}

\section{Decomposition of Antisymmetric Rank-2 Tensors}
We'll be working in Minkowski space with the metric
\begin{equation*}
    \eta = \text{diag}(+,-,-,-)
\end{equation*}

Consider the (arbitrary) rank-2 antisymmetric tensor $F^{\mu\nu}$. It may be uniquely expressed by the
following 3-vectors $\bm{E}$ and $\bm{B}$:
\begin{equation*}
\begin{split}
    F^{\mu \mu} &= 0 \\
    F^{0i} &= E^i \\
    F^{ij} &= -\epsilon^{ijk} B^k \\
\end{split}
\end{equation*}
The $F^{\mu\nu}$ tensor can then be written as:
\begin{equation*}
    F^{\mu\nu}
    =
    \begin{pmatrix}
        0 & E^x & E^y & E^z \\
        & 0 & -B^z & B^y \\
        & & 0 & -B^x \\
        & & & 0 \\
    \end{pmatrix}
\end{equation*}
where the lower half is the negative of the upper half. We may also write the vector components in terms
of the tensor:
\begin{equation*}
\begin{split}
    E^i &= F^{0i} \\
    B^i &= -\frac{1}{2} \epsilon^{ijk} F^{jk} \\
\end{split}
\end{equation*}
The signs and normalizations are chosen to match the familiar electromagnetic field tensor, electric fields,
and magnetic fields.

\textbf{Proof:}
The time-like field is trivially satisfied by these equations:
\begin{equation*}
    E^i = F^{0i} = E^i
\end{equation*}
And the space-like field:
\begin{equation*}
\begin{split}
    B^i
        &= -\frac{1}{2} \epsilon^{ijk} F^{jk} \\
        &= -\frac{1}{2} \epsilon^{ijk} \left( -\epsilon^{jkl} B^l \right) \\
        &= \frac{1}{2} (2\delta^{il}) B^l = B^i \\
\end{split}
\end{equation*}

\begin{prop}[Contractions of Antisymmetric Tensors Into Contractions of 3-Vectors]
    Consider two antisymmetric rank-2 $F^{\mu\nu}$ and $G^{\mu\nu}$ with decompositions into 3-vectors
    \begin{equation*}
    \begin{split}
        F^{0i} &= a^i \qquad \qquad G^{0i} = \alpha^i \\
        F^{ij} &= \epsilon^{ijk} b^k \qquad \qquad G^{ij} = \epsilon^{ijk} \beta^k \\
    \end{split}
    \end{equation*}
    Then the total contraction of these two tensors in terms of the three vectors is:
    \begin{equation*}
    \begin{split}
        F^{\mu\nu} G_{\mu\nu}
            &= 2 (a^i G_{0i} + b^k \frac{1}{2} \epsilon^{ijk} G_{ij}) \\
            &= -2(\bm{a} \cdot \bm{\alpha} - \bm{b} \cdot \bm{\beta}) \\
    \end{split}
    \end{equation*}
\end{prop}

\textbf{Proof:}
\begin{equation*}
\begin{split}
    F^{\mu\nu} G_{\mu\nu}
        &= F^{0i} G_{0i} + F^{i0} G_{i0} + F^{ij} G_{ij} \\
        &= 2 a^i G_{0i} + \epsilon^{ijk} b^k G_{ij} \\
        &= 2 (a^i G_{0i} + b^k \frac{1}{2} \epsilon^{ijk} G_{ij}) \\
        &= 2 (a^i (-\alpha^i) + b^k (\beta^k)) \\
        &= -2 (\bm{a} \cdot \bm{\alpha} - \bm{b} \cdot \bm{\beta}) \\
\end{split}
\end{equation*}

\textbf{Example:} Consider the generators of the Lorentz algebra
\begin{equation*}
    J^{\mu\nu} = i (x^\mu \partial^\nu - x^\nu \partial^\mu) 
\end{equation*}
which is an antisymmetric rank-2 tensor. This may be rewritten as two 3-vectors of generators
\begin{equation*}
\begin{split}
    L^i &= \frac{1}{2} \epsilon^{ijk} J^{jk} \\
    K^i &= J^{0i} \\
\end{split}
\end{equation*}
and the Lorentz generators in terms of the 3-vectors
\begin{equation*}
\begin{split}
    J^{0i} &= K^i \\
    J^{ij} &= \epsilon^{ijk} L^k \\
\end{split}
\end{equation*}
And so we have the component representation of the $J^{\mu\nu}$ tensor as
\begin{equation*}
    F^{\mu\nu}
    =
    \begin{pmatrix}
        0 & K^x & K^y & K^z \\
        & 0 & L^z & -L^y \\
        & & 0 & L^x \\
        & & & 0 \\
    \end{pmatrix}
\end{equation*}

\textbf{Example:} Consider the parameterization of the infinitesimal transformation of Lorentz transformations
in the 1/2-representation:
\begin{equation*}
    \Lambda_{1/2} = \exp\left( -\frac{i}{2} \omega_{\mu\nu} S^{\mu\nu} \right) \approx 1 - \frac{i}{2} \omega_{\mu\nu} S^{\mu\nu} 
\end{equation*}
Where $S^{\mu\nu}$ is an antisymmetric tensor
\begin{equation*}
    S^{\mu\nu} = \frac{i}{4} \left[ \gamma^\mu, \gamma^\nu \right] \\
\end{equation*}
Which in the Weyl basis we may decompose into
\begin{equation*}
\begin{split}
    B^i &= S^{0i} = \frac{i}{4}
        \begin{pmatrix}
            \sigma^i & 0 \\
            0 & -\sigma^i \\
        \end{pmatrix} \\
    \Sigma^i &= \epsilon^{ijk} S^{jk} \\
\end{split}
\end{equation*}
and $\omega_{\mu\nu}$ is a set of infinitesimal parameters, which we decompose into two 3-vectors as above:
\begin{equation*}
\begin{split}
    \omega^{0i} &= \beta^i \\
    \omega^{jk} &= \epsilon^{ijk} \theta^k \\
\end{split}
\end{equation*}
And so the infinitesimal transformation is:
\begin{equation*}
    \Lambda_{1/2} \approx 1 + i \left( \bm{\beta} \cdot \bm{B} - \bm{\theta} \cdot \bm{\Sigma} \right)
        = \exp \left( i(\bm{\beta} \cdot \bm{B} - \bm{\theta} \cdot \bm{\Sigma}) \right) \\
\end{equation*}

\end{document}
